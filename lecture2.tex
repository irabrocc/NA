% 声明为子文件,指定主文件
\documentclass[main.tex]{subfiles}

\begin{document}
\pagestyle{plain}
\setcounter{chapter}{1}
\chapter{Floating point numbers. Convergence. Bisection method. [Section 1.2]}
\label{chap:lecture2}
\subsection{Round-off errors and Computer arithmetic}
\par \noindent \textbf{Remark:}Computer don't operate with real numbers!
\par \noindent \textbf{Representations of decimal numbers(rational)}. 
\begin{enumerate}
    \item \textbf{Fixed Point} Let $d_k = \{0, 1, \cdots, 9\}$ Then we can represent a decimal number as
    \begin{equation}
        x = (d_5d_4d_3d_2d_1d_0.d_{-1}d_{-2}d_{-3}d_{-4}d_{-5}) = \sum_{k=-5}^{5} d_k 10^k
    \end{equation}
    , where $m< 0 \le M$. Similarly, for binary numbers: 
    \begin{equation}
        x = b_M b_{M-1} \cdots b_1 b_0 . b_{-1} b_{-2} \cdots b_{m} = \sum_{k=m}^{M} b_k 2^k, \quad b_k \in \{0, 1\}
    \end{equation}
    This is called the fixed-point representation. However, it's inconvenient for large numbers, or ones with many decimals, so it is rarely used(nowadays). 
    \item \textbf{Floating Point} Idea: represent the number as an integer scaled by an exponent of a fixed base. 
    \begin{equation}
        12.345 = (1)12345 \times 10^{-3}
    \end{equation}
    where $(1)$ is the sign bit, $12345$ is the significand(mantissa), $10$ is the base, and $-3$ is the exponent. 
    \par \noindent \textbf{Remark} Base $10$ corresponds to the "scientific notation used in calculators. Computers(usually) use base $2$. 
    \par \noindent \textbf{Remark} The common type of floating point representation follows the IEEE 754 standard. 
    \begin{enumerate}
        \item \textbf{Single precision} (binary 32, called float in C) 32 bits, mantisa 24 bits(approximately 7 decimal digits). 
        \item \textbf{Double precision} (binary 64, called double in C) 64 bits, mantisa 53 bits(approximately 16 decimal digits). 
    \end{enumerate}
\end{enumerate}

\begin{example}
    The first $33$ bits of $\pi$ in binary is
    \begin{equation}
        \pi = 110010010000111111011010101000100
    \end{equation}
    To write this as a single precision floating point number, take the 24 bit rounding approximation 
    \begin{equation}
        110010010000111111011011 = (1\times 2^0 + 1\times 2^{-1} + 0\times 2^{-2} + \cdots + 1\times 2^{-23})\times 2^1 \approx 3.1415928
    \end{equation}
    where the last bit is rounded up and the exponent is $1$.  
\end{example}
\par \textbf{Machine epsilon}(or machine precision) is an best bound for relative approximation error due to the floating point arithmetic. 
\par The standard values of machine epsilon are
\begin{itemize}
    \item binary 32(single precision): $\epsilon \approx 1.19 \times 10^{-7}$
    \item binary 64(double precision): $\epsilon \approx 2.22 \times 10^{-16}$
\end{itemize}
\par This value is contained in the module numpy of python (import numpy as np) 
\begin{lstlisting}[language = Python]
    import numpy as np
    print(np.finfo(float).eps)
    # 2.220446049250313e-16
    print(np.finfo(np.float32).eps)
    # 1.1920929e-07
\end{lstlisting}
\par \noindent \textbf{Remark} 
\begin{enumerate}
    \item "Machine epsilon accuracy" is the ultimate standard for numerical algorithms. (no better accuracy can be expected)
    \item Accuracy within $3$ or 4 decimals from the machine epsilon is already very sufficient and more than that cannot be expected in practice. 
\end{enumerate}

\begin{definition}
    Suppose $p*$ is an approximation of $p\in \mathbb{R}$. Then 
    \begin{enumerate}
        \item The \textbf{actual error} is $p - p*$;
        \item The \textbf{absolute error} is $|p - p*|$;
        \item The \textbf{relative error} is $\dfrac{|p - p*|}{|p|}$, provided $p\neq 0$.
    \end{enumerate}
    The number $p*$ is said to approximate $p\neq 0$ to $t$ significant digits if $t$ is the largest non-negative integer such that 
    \begin{equation}
        \dfrac{|p - p*|}{|p|} \le 5 \times 10^{-t} 
    \end{equation} 
\end{definition}
\par \noindent \textbf{Remark} It seems that there is a typo in the lecture note where it mistakes $\le$ for $<$ in the definition above. We have modified it. 

\begin{definition}[Floating Point Arithmetic]
    Denote by $fl(x)$ the floating point approximation of $x$. Then 
\begin{equation}
    x \oplus y = fl(fl(x) + fl(y))
\end{equation}
\begin{equation}
    x \otimes y = fl(fl(x) \cdot fl(y))
\end{equation}
\begin{equation}
    x \ominus y = fl(fl(x) - fl(y)) 
\end{equation}
\begin{equation}
    x \oslash y = fl\left(\dfrac{fl(x)}{fl(y)}\right), \quad y\neq 0 
\end{equation}
\end{definition}
\subsection{Convergence}
\begin{definition}[Rates of Convergence]
    Suppose $(\alpha_n), n = 0, 1, 2, \cdots $ is a sequence of real numbers such that $\lim_{n\to \infty} \alpha_n = \alpha\in \mathbb{R}$ and $(\beta_n)$ is a sequence such that $\lim_{n\to \infty} \beta_n = 0$. If there exist $K>0, N\ge 0$ such that $|\alpha_n - \alpha| \le K |\beta_n|$ for all $n \ge N$, then we say that $(\alpha_n)$ converges to $\alpha$ with rate $O(\beta_n)$. 
\end{definition}

\par \noindent \textbf{Remark} Usually $\beta_n = \dfrac{1}{n^p}$ for some $p>0$. 
\subsection{Bisection method}
\par Recall: Intermediate Value Theorem(Bolzano's theorem): Suppose $f:[a, b] \rightarrow \mathbb{R}$ is continuous, and there exists $K$ such that either $K\in (f(a), f(b))$ if $f(a) < f(b)$ or $K\in (f(b), f(a))$ if $f(a) > f(b)$. Then there exists $c\in (a, b)$ such that $f(c) = K$. As a corollary, if $f(a)f(b)<0$, then there exists $c\in (a, b)$ such that $f(c) = 0$. 
\par \noindent \textbf{Application} Finding an interval $[a, b]$ that contains a solution of $x - 2^{-x} = 0$. For $x = 0, f(0) = -1 < 0$. For $x = 1, f(1) = 0.5 > 0$. So by the corollary there exists $c\in (0, 1)$ such that $f(c) = 0$.
\par The following is an more advanced application. 
\begin{algorithm}[Bisection Method]
    Let $f:[a, b] \rightarrow \mathbb{R}$ be continuous and suppose that $f(a)f(b) < 0$. Then
    \begin{enumerate}
        \item Set $n = 1, a_1 = a, b_1 = b$.
        \item Compute $c_n = \dfrac{a_n + b_n}{2}$. If $f(c_n) = 0$, stop. Otherwise, go to step 3.
        \item If $f(a_n)f(c_n) < 0$, set $a_{n+1} = a_n, b_{n+1} = c_n$. If $f(b_n)f(c_n) < 0$, set $a_{n+1} = c_n, b_{n+1} = b_n$. Increment $n$ by $1$ and go to step 2.
    \end{enumerate}
\end{algorithm}

\begin{example}
    Let $f(x) = \cos(x) - 2x, [a, b] = [-8, 10]$. Then $f(-8) > 0, f(10) < 0$. So we can apply the bisection method.
\end{example}
\begin{theorem}
    Suppose $f\in C([a, b])$ and $f(a) f(b) < 0$. Then the sequence $(c_n)$ generated by the bisection method approximates the zero $p\in (a, b)$ of $f$ with  
    \begin{equation}
        |c_n - p| \le \dfrac{b - a}{2^{n}}, n \ge 1 
    \end{equation}
\end{theorem}
\par \noindent \textbf{Proof} Let $n \ge 1$. Then $b_n - a_n = \dfrac{b - a}{2^{n-1}}$ and $p\in (a_n, b_n)$ where $a_n, b_n$ are the endpoints of the interval at the $n$-th step. Because $c_n = \dfrac{a_n + b_n}{2}$, it follows that
\begin{equation}
    |c_n - p| \le \dfrac{b_n - a_n}{2} = \dfrac{b - a}{2^n} \to 0  
\end{equation}
as $n\to \infty$. 
\\ \null \hfill $\blacksquare$ 

\par \noindent \textbf{Remark} As $|p_n - p| \le (b - a) 2^{-n}$. So the convergence converges with the rate $O(2^{-n})$. Note: Here we can take $K = b - a$.
\par \noindent \textbf{Remark} To avoid taking off points by graders, we're expected to show(at lest state) the conditions of the theorems we use. For example we need to say $K = b - a $ as in the previous proof of the theorem. 
\end{document}